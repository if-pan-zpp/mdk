\documentclass{article}
\usepackage{geometry}
\usepackage{esvect}
\usepackage{amsmath}
\usepackage{amsfonts}
\usepackage{mathtools}
\usepackage{bm}
\usepackage{lmodern}
\usepackage{lipsum}

\DeclarePairedDelimiter{\p}{(}{)}
\DeclarePairedDelimiter{\norm}{\lvert}{\rvert}
\DeclarePairedDelimiter{\normed}{\lVert}{\rVert}
\DeclarePairedDelimiter{\inner}{\langle}{\rangle}
\newcommand{\R}{\mathbb{R}}
\newcommand{\statediff}[1]{\frac{\partial #1}{\partial r_n^\mu}}
\newcommand{\cross}{\times}

\title{Forcefield}
\author{}
\date{}

\begin{document}
  \maketitle

  \section*{Introduction}
  Notation:
  \begin{enumerate}
    \item $r_i$ -- the position of the particle; $v_i = r_{i+1} - r_i$ is the vector from $i$th to $(i+1)$th particle;
    \item $t$ -- time;
    \item $V$ -- the potential of the force field;
    \item $v^\mu$ -- the $\mu$th component of $v$;
    \item $D$ -- an arbitrary 1d differential operator, for example $\partial/\partial t$;
    \item $\nabla_n^\mu = \partial/\partial (r_n)^\mu$, in particular of interest is $(F_n)^\mu = -\nabla_n^\mu V$;
    \item $e$ -- the vector of ones;
    \item $\varepsilon_{ijk}$ -- the Levi-Civita tensor;
    \item $\delta_{\mu\nu}$ -- the Kronecker delta;
    \item $\norm{v}$ -- the norm, $\normed{v} = v/\norm{v}$ -- the normalized vector.
  \end{enumerate}

  Calculus hacks:
  \begin{align*}
    D\norm{f} &= D\sqrt{f \circ f} = \frac{Df \cdot f + f \cdot Df}{2\sqrt{f \cdot f}} = \frac{f \cdot Df}{\norm{f}}\\
    D\normed{f} &= D\frac{f}{\norm{f}} = \frac{\norm{f}Df-f D\norm{f}}{\norm{f}^2} = \frac{\norm{f}^2 Df - (f \cdot Df)f}{\norm{f}^3}
  \end{align*}

  Vector algebra hacks:
  \begin{align*}
    D(f \cdot g) &= (Df) \cdot g + f \cdot Dg\\
    D(f \cross g) &= (Df) \cross g + f \cross Dg\\
    (a \cross b) \cdot (c \cross d) &= (a \cdot c)(b \cdot d) - (a \cdot d)(b \cdot c)\\
    (p_0 \cross p_1) \cdot p_2 &= \varepsilon_{ijk} (p_i \cross p_j) \cdot p_k\\
    (a \cross b) \cross c &= (a \cdot c)b - (a \cdot b)c
  \end{align*}

  \section*{Harmonic tethers}
  The formula for the potential between residues $i$ and $j$ is:
  \begin{align*}
    V &= \frac{1}{2}k_2 (\norm{v_{ij}}-d_0)^2 + \frac{1}{4}k_4 (\norm{v_{ij}}-d_0)^4\\
    DV &= \p*{k_2 (\norm{v_{ij}}-d_0) + k_4 (\norm{v_{ij}}-d_0)^3} D\norm{v_{ij}}
  \end{align*}
  where $k_2$ and $k_4$ are harmonic and anharmonic parts, and
  \begin{align*}
    \nabla_i^\mu \norm{v_{ij}} &= \frac{v_{ij} \cdot \nabla_i^\mu v_{ij}}{\norm{v_{ij}}} = -\frac{v_{ij}^\mu}{\norm{v_{ij}}}\\
    \nabla_j^\mu \norm{v_{ij}} &= -\nabla_i \norm{v_{ij}}
  \end{align*}

  In the original code, $k_2$ is {\tt H1}, and $k_4$ is {\tt HH1}.

  \section*{Bond angle}
  The bond angle between $i$, $j$ and $k$ is $\theta$, where:
  \begin{align*}
    \cos{\theta} &= \frac{v_{ji} \cdot v_{jk}}{\norm{v_{ji}}\norm{v_{jk}}}\\
    \sin{\theta} &= \frac{\norm{v_{ji} \cross v_{jk}}}{\norm{v_{ji}}\norm{v_{jk}}}
  \end{align*}
  In general,
  \begin{align*}
    D\cos{\theta} = -\sin{\theta} D\theta \implies D\theta = -\frac{D\cos{\theta}}{\sin{\theta}}
  \end{align*}
  Let's start with $i$:
  \begin{align*}
    \nabla_i^\mu \cos{\theta} &= \nabla_i^\mu (\normed{v_{ji}} \cdot \normed{v_{jk}})\\
    &= \normed{v_{jk}} \cdot \nabla_i^\mu \normed{v_{ji}}\\
    &= \normed{v_{jk}} \cdot \frac{\norm{v_{ji}}^2 e^\mu - v_{ji}^\mu v_{ji}}{\norm{v_{ji}}^3}\\
    &= \frac{\norm{v_{ji}}^2 v_{jk}^\mu - (v_{ji} \cdot v_{jk})v_{ji}^\mu}{\norm{v_{jk}}\norm{v_{ji}}^3}\\
    \nabla_i \cos{\theta} &= \frac{v_{ji} \cross (v_{ji} \cross v_{jk})}{\norm{v_{jk}}\norm{v_{ji}}^3}
  \end{align*}
  and similarly:
  \begin{align*}
    \nabla_k \cos{\theta} &= \frac{v_{jk} \cross (v_{jk} \cross v_{ji})}{\norm{v_{ji}}\norm{v_{jk}}^3}
  \end{align*}
  The derivative for $j$ should be symmetric. Thus:
  \begin{align*}
    \nabla_i \theta &= -\frac{v_{ji} \cross (v_{ji} \cross v_{jk})}{\norm{v_{ji}}^2 \norm{v_{ji} \cross v_{jk}}}\\
    \nabla_j \theta &= -\nabla_i \theta -\nabla_k \theta\\
    \nabla_k \theta &= -\frac{v_{jk} \cross (v_{jk} \cross v_{ji})}{\norm{v_{jk}}^2 \norm{v_{ji} \cross v_{jk}}}\\
  \end{align*}

  As for the potentials, there are three:
  \begin{enumerate}
    \item from the native structure:
    \begin{align*}
      V &= k_\theta (\theta - \theta^{nat})^2\\
      DV &= 2k_\theta (\theta - \theta^{nat}) D\theta 
    \end{align*}

    \item heurestic -- in the form of a sixth degree polynomial:
    \begin{align*}
      V &= \sum_{d=0}^D \alpha_d \theta^d\\
      DV &= \p*{\sum_{d=1}^D d\alpha_d \theta^{d-1}}D\theta
    \end{align*}

    \item tabulated:
    \begin{align*}
      V &= f(\theta)\\
      DV &= (Df)(\theta) D\theta
    \end{align*}
  \end{enumerate}

  In the original code, $\theta_i$ denotes the bond angle between $i-1$, $i$ and $i+1$ (this is important for loading sequence definitions and contact maps). $k_\theta$ is {\tt CBA}.

  \section*{Dihedral angle}
  The dihedral angle between $i$, $j$, $k$ and $l$ is $\phi$, where:
  \begin{align*}
    \cos{\phi} &= \frac{(v_{ij} \cross v_{jk}) \cdot (v_{jk} \cross v_{kl})}{\norm{v_{ij} \cross v_{jk}} \norm{v_{jk} \cross v_{kl}}}\\
    \sin{\phi} &= \frac{\norm{v_{jk}}v_{ij} \cdot (v_{jk} \cross v_{kl})}{\norm{v_{ij} \cross v_{jk}} \norm{v_{jk} \cross v_{kl}}}
  \end{align*}

  We have:
  \newcommand{\crossA}{v_{ij} \cross v_{jk}}
  \newcommand{\crossB}{v_{jk} \cross v_{kl}}
  \begin{align*}
    \nabla_i^\mu \cos{\phi} &= \normed{\crossB} \cdot \nabla_i^\mu \normed{\crossA}\\
    \nabla_i^\mu \normed{\crossA} &= \frac{\norm{\crossA}^2\nabla_i^\mu(\crossA) - (\crossA \cdot \nabla_i^\mu(\crossA))\crossA}{\norm{\crossA}^3}\\
    &= \frac{\norm{\crossA}^2 (-e^\mu \cross v_{jk}) - ((\crossA) \cdot (-e^\mu \cross v_{jk}))\crossA}{\norm{\crossA}^3}
  \end{align*}
\end{document}