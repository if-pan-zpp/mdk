\documentclass{article}
\usepackage{geometry}
\usepackage{esvect}
\usepackage{amsmath}
\usepackage{amsfonts}
\usepackage{mathtools}
\usepackage{bm}
\usepackage{lmodern}
\usepackage{lipsum}

\DeclarePairedDelimiter{\brk}{(}{)}
\DeclarePairedDelimiter{\norm}{\lvert}{\rvert}
\DeclarePairedDelimiter{\normed}{\lVert}{\rVert}
\DeclarePairedDelimiter{\inner}{\langle}{\rangle}
\newcommand{\R}{\mathbb{R}}
\newcommand{\statediff}[1]{\frac{\partial #1}{\partial r_n^\mu}}
\newcommand{\cross}{\times}

\title{Forces}
\author{}
\date{}

\begin{document}
  \maketitle

  \section*{Notation}
  Let:
  \begin{enumerate}
    \item $r_i$ be the position of the particle;
    \item $t$ time;
    \item $V$ the potential of the force field;
    \item $x^\mu$ the (0-indexed) $\mu$th component of vector $x$;
    \item $\partial_n^\mu = \partial/\partial \brk*{r_n}^\mu$ -- in particular of interest is $\brk*{F_n}^\mu = -\partial_n^\mu V$;
    \item $e_\mu$ the unit vector for $\mu$th component;
    \item $v_i = r_{i+1}-r_i$;
    \item $\varepsilon_{ijk}$ the Levi-Civita tensor;
    \item $\norm{v}$ the norm, $\normed{v} = v/\norm{v}$
  \end{enumerate}

  \section*{Algebra recap}
  \begin{align*}
    \frac{\partial f \cdot g}{\partial x} &= \frac{\partial f}{\partial x} \cdot g + f \cdot \frac{\partial f}{\partial x}\\
    \frac{\partial f \cross g}{\partial x} &= \frac{\partial f}{\partial x} \cross g + f \cross \frac{\partial g}{\partial x}\\
    \brk*{p_0 \cross p_1} \cdot p_2 &= e_{ijk} \brk*{p_i \cross p_j} \cdot p_k\\
    v \cdot e_\mu &= v^\mu
  \end{align*}

  \section*{Chirality}
  \begin{align*}
    V &= \frac{1}{2} e_{chi} \sum (C_i - C_i^{nat})^2\\
    C_i &= \frac{\brk*{v_{i-1} \cross v_i} \cdot v_{i+1}}{d_0^3}
  \end{align*}
  where $d_0$ is $|v_{i-1}|$ in the native state, and $C_i^{nat}$ is $C_i$ in the native state.

  \begin{align*}
    \partial_n^\mu V &= e_{chi} \sum (C_i - C_i^{nat}) \partial_n^\mu C_i\\
    \partial_n^\mu C_i &= \frac{1}{d_0^3} \partial_n^\mu \brk{\underbrace{\brk*{v_{i-1} \cross v_i} \cdot v_{i+1}}_{f_i}}
  \end{align*}

  with nonzero values:
  \begin{align*}
    \partial_{i-1}^\mu f_i &= \brk*{-\varepsilon^\mu \cross v_i} \cdot v_{i+1}\\
    &= -\brk*{v_i \cross v_{i+1}}^\mu\\
    \partial_i^\mu f_i &= \brk*{e^\mu \cross v_i - v_{i-1} \cross e^\mu} \cdot v_{i+1}\\
    &= \brk*{v_i \cross v_{i+1} + v_{i-1} \cross v_{i+1}}^\mu\\
    \partial_{i+1}^\mu f_i &= \brk*{v_{i-1} \cross e^\mu} \cdot v_{i+1} + \brk*{v_{i-1} \cross v_i} \cdot \brk*{-e^\mu}\\
    &= -\brk*{v_{i-1} \cross v_{i+1} + v_{i-1} \cross v_i}^\mu\\
    \partial_{i+2}^\mu f_i &= \brk*{v_{i-1} \cross v_i}^\mu
  \end{align*}

  \section*{Harmonic tethers}
  We have:
  \begin{align*}
    V &= \sum \brk*{\frac{1}{2}k_2 \norm{v_i}^2 + \frac{1}{4}k_4 \norm{v_i}^4}\\
    \partial_n^\mu V &= \sum \brk*{k_2 \norm{v_i} + k_4 \norm{v_i}^3} \partial_n^\mu \norm{v_i}
  \end{align*}
  where
  \begin{align*}
    \partial_n^\mu \norm{v_i} &= \partial_n^\mu \sqrt{v_i \cdot v_i} = \frac{v_i \cdot \partial_n^\mu v_i}{\norm{v_i}}\\
    \partial_i^\mu \norm{v_i} &= \frac{v_i \cdot \brk*{-e_\mu}}{\norm{v_i}}= -\normed{v_i}^\mu\\
    \partial_{i+1}^\mu \norm{v_i} &= \normed{v_{i+1}}^\mu
  \end{align*}

  \section*{Bond angle}
  We define $\theta_i$ as the bond angle between $i-1, i, i+1$:
  \begin{align*}
    &\cos \theta_i = \frac{v_{i-1} \cdot v_i}{\norm*{v_{i-1}} \norm*{v_i}}
  \end{align*}

  We have:
  \begin{align*}
    \partial_n^\mu \normed{f} &= \partial_n^\mu \frac{f}{\sqrt{f \cdot f}} = \frac{\partial_n^\mu f |f| - f \partial_n^\mu \sqrt{f \cdot f}}{f \cdot f}\\
    &= \frac{1}{f \cdot f} \brk*{|f|\partial_n^\mu f - f \frac{1}{\sqrt{f \cdot f}}\brk*{f \cdot \partial_n^\mu f}}\\
    &= \frac{\partial_n^\mu f}{\norm{f}} - \frac{f \cdot \partial_n^\mu f}{\norm{f}^3}\\
    \partial_i^\mu \normed{v_i} &= -\frac{e_\mu}{\norm{v_i}} + \frac{f^\mu}{\norm{v_i}^3}\\
    \partial_{i+1}^\mu \normed{v_i} &= \frac{e_\mu}{\norm{v_i}} - \frac{f^\mu}{\norm{v_i}^3}
  \end{align*}

  and so:
  \begin{align*}
    \partial_n^\mu \theta_i &= \partial_n^\mu \arccos \brk*{\normed{v_{i-1}} \cdot \normed{v_i}}\\
    &= \frac{\partial_n^\mu \normed{v_{i-1}} \cdot \normed{v_i} + \normed{v_{i-1}} \cdot \partial_n^\mu \normed{v_i}}{\sqrt{1-\brk*{\normed{v_{i-1}} \cdot \normed{v_i}}^2}}
  \end{align*}

  with nonzero values:
  \begin{align*}
    \partial_{i-1}^\mu \theta_i &= \frac{1}{\sqrt{1-\brk*{\normed{v_{i-1}}\cdot \normed{v_i}}^2}}\brk*{\brk*{-\frac{e_\mu}{\norm{v_{i-1}}} + \frac{f^\mu}{\norm{v_{i-1}}^3}}\cdot \normed{v_i}}\\
    \partial_{i}^\mu \theta_i &= \frac{1}{\sqrt{1-\brk*{\normed{v_{i-1}}\cdot \normed{v_i}}^2}}\brk*{\brk*{-\frac{e_\mu}{\norm{v_i}} + \frac{f^\mu}{\norm{v_i}^3}}\cdot \normed{v_i}+\normed{v_{i-1}} \cdot \brk*{\frac{e_\mu}{\norm{v_i}} - \frac{f^\mu}{\norm{v_i}^3}}}\\
    \partial_{i+1}^\mu \theta_i &= \frac{1}{\sqrt{1-\brk*{\normed{v_{i-1}}\cdot \normed{v_i}}^2}}\brk*{\normed{v_i} \cdot \brk*{\frac{e_\mu}{\norm{v_{i+1}}} - \frac{f^\mu}{\norm{v_{i+1}}^3}}}
  \end{align*}

  Native bond angle potential is:
  \begin{align*}
    V &= k_\theta \sum \brk*{\theta_i - \theta_i^{nat}}^2\\
    \partial_n^\mu V &= 2k_\theta \sum \brk*{\theta_i - \theta_i^{nat}}\partial_n^\mu \theta_i
  \end{align*}

  A heurestic bond angle is a polynomial dependent on residue types in the chain, i.e.
  \begin{align*}
    V &= \sum_i \sum_{d=0}^D a_{i,d} \theta_i^d\\
    \partial_n^\mu V &= \sum_i \sum_{d=1}^D a_{i,d} d\theta_i^{d-1} \partial_n^\mu \theta_i
  \end{align*}

  A tabulated bond angle is:
  \begin{align*}
    V &= \sum f(\theta_i)\\
    \partial_n^\mu V &= \sum \brk*{\partial_n^\mu f}(\theta_i) \partial_n^\mu \theta_i 
  \end{align*}

  \section*{Dihedral angles}
  Dihedral angle $\phi_i$ between residues $i-2$, $i-1$, $i$ and $i+1$ is defined as:
  \begin{align*}
    \cos{\phi_i} &= \frac{\brk*{v_{i-1}\cross v_i}\cdot\brk*{v_i\cross v_{i+1}}}{\norm{v_{i-1}\cross v_i}\norm{v_i\cross v_{i+1}}}
  \end{align*}

\end{document}